% book of calculus of variations exercises
% nicola frosi
% starting date 18-05-2024

\documentclass[a4paper, twoside, openany]{book}

\usepackage[utf8]{inputenc}
\usepackage[T1]{fontenc}
\usepackage[english]{babel}

\usepackage[margin=2.5 cm]{geometry}

\usepackage{float}
\usepackage{rotating}


\usepackage{amsmath}
\usepackage{amsfonts}
\usepackage{amssymb}
\usepackage{amsthm}

\usepackage{tikz}
\usetikzlibrary{arrows.meta, calc, quotes}

\title{\textbf{\huge{\textit{A collection of Calculus of Variations exercises}}}}

\begin{document}
\maketitle
\chapter{Scalar Case}
\section*{Exercise $1$}
Let
$$I[u] = \int_a^b 2 u'(x)^3 dx.$$
Find a certain function $u: [a, b] \rightarrow \infty$ that is a minimum of the given integral functional. 
\section*{Solution}
$$F(x, u(x), u'(x)) = F(u'(x)) = 2 u'(x)^3.$$
We can rewrite the integrand as
$$F(u'(x)) = F(p) = 2 p^3$$
From the Euler-Lagrange equations:
$$\frac{d}{dx}\frac{\partial F}{\partial u'} (u'(x)) = \frac{\partial F}{\partial p}(p) = \frac{\partial F}{\partial u}(p)$$
we have
$$\frac{d}{dx}\frac{\partial F}{\partial p} (p) = 0.$$
Since
$$\frac{\partial F}{\partial p} (p) = \frac{\partial (2 p^3)}{\partial p} = 6 p^2 = 6 u'(x)^2$$
then
$$\frac{d}{dx}\frac{\partial F}{\partial u'}(u'(x)) = 12 u'(x) u''(x).$$
We obtain an ordinary differential equation of the second order
$$12 u'(x) u''(x) = 0$$
$$u'(x) = C_1$$
$$u(x) = C_2 x.$$
We now assume $a = 1$ and $b = 10$.
$$\begin{cases}
	u(x) = C_2 x \\
	u'(x) = C_1 \\
	u(1) = 2 \\
	u'(10) = 4
\end{cases}$$
$$C_1 = 4$$
$$C_2 = 2$$
$$u(x) = 2 x$$
\begin{figure}[!ht]
\begin{center}
\begin{tikzpicture}[scale=0.5]
\draw[->] (-0.5,0)--(13.0,0) node[below]{$x$};   
\draw[->] (0,-0.5)--(0,22.5)  node[left]{$y$};
\draw[dashed] (1.0,-1.5)--(1.0,22.5);
\draw[dashed] (10.0,-1.5)--(10.0,22.5);
\path
(0,0) node[below left]{$0$};
\path
(1.0,0) node[below right]{$1.0$};
\path
(10.0,0) node[below left]{$10.0$};
\path
(10.0,5.0) node[below right]{$u(x)$};
\foreach \i in {0,0.5, 1.0,...,12.0} \draw (\i,-0.05)--(\i,0.05);
\draw[red, domain=1.:10., samples=100, variable=\x] plot ({\x}, {2.*\x});
\end{tikzpicture}
\end{center}
\end{figure}
\clearpage

\end{document}